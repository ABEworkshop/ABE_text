\section{State-of-the-Art}
%In this section,we attempt to survey the existing literature and constructions for ABE schemes over the period 2005-2016 from the aspects of three trunk clues,function,efficiency and security.Then we explain how do we arrive at more  logical and clear clues on the development of ABE.\subsection{the development Of ABE}
\subsection{the birth of ABE (by Lei Xu)}
Attribute-based encryption originates from fuzz identity based encryption \cite{Sahai:EUROCRYPT'05} by Amit Sahai and Brent Waters. %this idea Hierarchical identity-based encryption schemes\cite{Danfeng:CCS'04}and
Fuzz identity based encryption supports decrypting by the attributes with selectable lower-limit number,
due to use of the threshold secret sharing technology \cite{Shamir:CACM'79}.
After that, beyond the threshold policy, much attention on general policy has been paid.
The general policy can tie data owner with data user by a series of formalized constraint,
which supports general relationship among attributes.
%point-to-point constraint(corresponding the privilege management infrastructure technology)
Access control technology just performs well,
and attribute-based encryption has perfectly formed only when this technology are put into use.
More specifically, by using access control technology, ABE inherently includes two types of manners, KP-ABE and CP-ABE.
Goyal et al. \cite{Goyal:CCS'06} propose the first KP-ABE scheme,
while John Bethencourt et al. \cite{Bethencourt:SP'07} constructs the first CP-ABE scheme.
These two schemes concretely achieve the access control,
so we contend that they sign the birth of ABE.
Since then, the researches on ABE tend to be booming,
and we summarize the trunk directions of development on ABE,
i.e, function, efficiency, security and expressions on access control.
\subsection{efficiency (by Hao Zhang)}
ABE brings a new idea of encryption. Whereas in today's life it is not widely used, the traditional ABE encryption and decryption speed can not make people satisfied in nowadays big data situation on the Internet. Based on this dilemma, many new ABE schemes are proposed to improve the efficiency of ABE. The first is ABE with constant ciphertext length. The length of the ciphertext depends on the number of attributes in previous ABE schemes. Keita Emura \cite{Keita:ISPEC'09} first put forward the ABE with constant ciphertext length, unlike the previous ABE schemes they put the user attributes in one set, the whole set is used as the generation of ciphertext, so that fixed the ciphertext length. Then Javier \cite{Javier:PKC'10} propose the collusion-resistant ABE scheme with constant ciphertext length. Even if the ABE with constant ciphertext length is not secure enough, thus in 2013 Nishant Doshi \cite{Nishant:Wiley'14} proposed a fully secure ABE with constant ciphertext length, that makes the ABE with constant ciphertext length can be in the security with a higher efficiency.
	However, in order to improve the efficiency of the ABE scheme, not only the constant ciphertext length of this scheme. Fuchun Guo \cite{Fuchun:TIFS'14} in 2014 proposed a constant keys length rather than the ciphertext length of the ABE scheme makes ABE scheme can be applied to lightweight devices. In addition to constant ciphertext and constant keys way, there is another way to improve efficiency. Luan Ibraimi \cite{Luan:ISPEC'09} proposed a highly efficient CP-ABE scheme with "?" and "?" access structure. By operating on the access structure, decryption is used only once to improve efficiency. Nuttapong Attrapadung \cite{Tadanori:ACNS'16} proposed the size of the ciphertext and the size of the key tradeoff scheme can be carried out directly to improve the efficiency of ABE scheme.
	In the ABE scheme there is a common method to enhance efficiency, that is outsource. The ABE program outsourcing the complex encryption process and decryption process outsourcing to the cloud server.Let high-performance server to help us to compute the most complicated part of the scheme, and we only accept the results of the final operation out.

\subsection{security (by Yi Wang)}
1:Survey on original ABE,and some articles,which have promoted in security,based on original ABE.

2:You describe these articles in time sequence.

3:Also need a table to compare security among articles mentioned by you.

\subsection{function (by Lei Xu)}
Although original ABE schemes are of great significance as mentioned in our introduction,
there are still some limitation for some specific scenario.
For example, the principal of the university wants to send a encrypted file to those who satisfy following constraint:
((``school: EECS" and ``career: teacher") or (``career: administrator")) and ``age: 35-45".
Obviously, original ABE schemes are not able or inefficient to deal with above situation.
So, some schemes with additional functions are proposed in order to support the various applications.

In this subsection, we survey the development on additional function compared with original ABE schemes.
First constraint in original ABE schemes is that each attribute only has two state: a attribute is held by somebody or not.
Chun-I Fan et al.\cite{Fan:TOC'14} straightforward propose the Arbitrary-State Attribute-Based Encryption in 2014.
In \cite{Fan:TOC'14}, ``teacher'' and ``administrator'' are two values of the only one attribute ``career" in our example,
i.e, the scheme has the function supporting attribute with many values.
This kind function leads to the fact that the number of designed attributes somebody receives is comparable to the number of natural attributes it has.
In fact, in 2009,
Rakesh Bobba et al.\cite{Bobba:ESORICS'09} have achieved multiple value assignment on an attribute, which arbitrary-state attribute is somewhat similar to.

Additionally, ``numerical attribute", ``age" in our instance, means that the value of an attribute is numerical, i.e,
we can compare those values using operations such us $\leq, >, \neq$ and so on.
In the original CP-ABE scheme\cite{Bethencourt:SP'07}, ``numerical attribute" and the operation on values has been posed,
but not applied in the concrete scheme.
Lang, Bo et al.\cite{Lang:SC'13} in 2013 achieves the construct of an ABE scheme supporting ``numerical attribute" and the operation on its values.
Having the function supporting ``numerical attribute" and the operation on values, ABE schemes can achieve more
effective data self-protection mechanisms in open environments such as Cloud computing \cite{Lang:SC'13}.

Moreover, Weighted Attribute Based Encryption is proposed by Ximeng Liu et al.\cite{Ximeng:ICC'14} to deal with the situation that in some circumstances the attributes are not always in the same position, i.e,
different attributes have different importance.

%\subsection{clues on the development of ABE}
%The surveys above clearly display the development of ABE due to the three main clues obtained.The reveal of how we find out more detail clues is shown in this subsection.In brief,focus on the purposes and process of all or most present ABE schemes.The detailed procedures are explained as followings.Firstly,pay our attention to the all parts of process on original ABE schemes i.e\cite{classical:KP-ABE} and \cite{first:CP-ABE},and draw a picture containing three parts:components of specific scheme,participants and operations of process of these schemes,shown in Fig.~\ref{fig:original_clues}.We note that operations are actually not independent to components and participants,this division is just for discerning more clearly,and that mentioned clues above don't conclude access structure,so this method can also be used in other public-key encryption schemes.Secondly,we ligature each two parts which there are some responding relationships between,then the lines are labeled with relationships.For example,there is an "authorization" between "CA" and "user",so we line this two parts and label with "authorization".After finished lines of each existing relations,a new issue of ABE is added in this kind figure in the way concentrate attention to the purpose and the process.For example,constant ciphertext issue of ABE can be add as Fig.~\ref{fig:cons_ciphertext_added_clues},and proxy re-encryption ABE can be added as Fig.~\ref{fig:proxy_added_clues}.Finally,after surveyed the most issues of ABE,we obtain the Fig.~\ref{fig:final_clues}.All lines in Fig.~\ref{fig:final_clues} is called as detailed clues,We can clearly view the motivations or main process of present ABE schemes using our clues,so that a overview of ABE tend to be crystallized and visual.Besides,researchers also can get some possible directions of exploration of ABE though using this kind way to line parts which haven't lined in Fig.~\ref{fig:final_clues}.For example,we can construct called common constraint ABE(CCABE) through lining between plaintext and encryption.In CCABE,universal attributes is defined by some trusted authority.These attributes are divided into some groups,and each group contains arbitrary number attributes.Then according to the distributed attribute group every data owner holds,every owner in this system can define its own access policy which it wants user's attributes to satisfy in order to decrypt.And only when users attributes satisfy all or some other rules(such us threshold policy)from data owners' access policy.
%%\begin{figure}
%%  \centering
%%    \includegraphics[width=0.7\columnwidth]{fig/clues.eps}
%%  \caption{clues on the development of ABE,that's from the all parts }\label{fig:clues}
%%\end{figure}
