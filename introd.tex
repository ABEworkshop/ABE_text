\section{introduction}
Attribute-based encryption(ABE) play an important role in encryption scheme of cloud such us\cite{Li Ming:tpds},\cite{other:12},\cite{kp:11},\cite{policy:16},\cite{other:20} due to that ABE is not only powerful mechanism for protecting the confidentiality of stored and transmitted information inheriting traditional public-key encryption but also achieves the stride from one-to-one model from previous encryption to one-to-many\cite{cp:expressive},and specifically implementing access control and supporting storing in untrusted storage server are the reasons for its achieving.Additionally,ABE has following concrete advantages:\subsubsection{High Efficiency}:The cost of encryption and decryption is only related with the length of ciphertext and the number of attributes,and has nothing to do with the number of users;\subsubsection{Dynamic}:decryption can be completed by user only when the attributes satisfy the policy,and has nothing to do with the time when the user join this systems.\subsubsection{Flexibility}:ABE supports various access structure such us threshold policy and boolean format.\subsubsection{Privacy}:when data owner encrypt,he or she has no need to know the specific identities of individuals who will authorised decrypt\cite{Feng:survey}.

ABE is actually an extension of identified based encryption(IBE) and it can regard as a special IBE that multiply identities which meet a specific constraint can decrypt the ciphertext.In ABE,that constraint is just the access policy,and user can decrypt only when attributes satisfy the policy.For integrity,there is a simple division,which is only helpful on the application context and make little sense for the more heuristic research on ABE after it has formed,of ABE into ciphertext-policy attribute-Based encryption(CP-ABE) and key-policy attribute-Based encryption(KP-ABE).For CP-ABE,user keys are associated with sets of attributes,whereas ciphertexts are associated with policies,and for KP-ABE,the case is the contrary.

With the development of heuristic research on ABE,more and more ABE variants has been growing.However,there has been no integrated overview published that should not just concludes most of topics of ABE but also figure out the logical and clear clues or veins of development of ABE from pre-birth to now.\cite{IE:survey}does the survey on ABE of access control,%but CP-ABE,KP-ABE,ABE Scheme with Non-Monotonic Access Structures,and Hierarchical ABE are parallel status from the article structure,which means those kind of schemes belong to the same category in logic.
but just enumerate CP-ABE,KP-ABE,ABE Scheme with Non-Monotonic Access Structures,and Hierarchical ABE so from this article we can't find out the relationship among those schemes.From the aspect of difficulties of ABE,\cite{Su:survey}
analyzes the access structure,attribute revocation,key-abuse problem and multi-authority scheme,but those issues don't contain all of that in ABE.\cite{Feng:survey} points out the direction of development on concrete topics of ABE,however there is also no clear hierarchy of ABE schemes.

\subsection{Motivation And Contributions}
When everything occurs to us,the ultimate questions of which we would like to know the answer are always about past,now,and future.Now those happen to ABE.Detailedly,the three questions are:
\begin{itemize}
\item Where does ABE come from?
\item What��s the essence of ABE?and how is it going?
\item where will ABE go?
\end{itemize}
Our motivation is finding out the answers of these three questions.

As for our contributions,the first is figuring out the logical and clear clues of the development of ABE based on present ABE schemes,and the clues are helpful to that researchers on ABE can follow specific directions to more exploration and that related others can get clear and integrated general view of ABE.Then we propose a taxonomy which contains all or most ABE schemes,and the taxonomy can be used by researchers to answer many important questions such us:
\begin{itemize}
\item what's the reasons for emergences of those kinds of ABE schemes?
\item what's the relationship among ABE schemes?
\item what's the more possible work along developed directions of ABE?
\end{itemize}
Additionally,this paper explores the design philosophy of classical ABE schemes,which can be a reference for researchers when he or she wants to find the principles of some schemes.After all above,we propose some foresight about the future of ABE.

\subsection{Organization}
In this paper,we survey some attribute-based encryption schemes,and propose clues and a taxonomy of the development of present ABE schemes and give some opinions about the future directions of ABE. The organization of the paper is organized as follows.In section 2,we review the key nodes of development of ABE,then draw its clues.Syntax of ABE schemes is introduced in section 3.In section 4,we consider the design philosophy of two classical ABE schemes.Access control,the most essential technology in ABE,are catched from most present ABE schemes in section 5.We propose a taxonomy of the development of present ABE schemes in section 6,and this taxonomy implies all or most direction along which ABE has developed.In section 6,we do some comparisons,each comparison is made among schemes in a same category,and these targets,used by comparison,are just these purposes of schemes in the same category.extension in section 7 is about several interdiscipline that hasn't been ripe so far,but that has the possibility to be a part of ABE as mature technologies used.
