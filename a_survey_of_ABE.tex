
\documentclass[journal]{IEEEtran}


\usepackage{amsmath}
\usepackage{amsfonts}
\usepackage{epsfig}
\usepackage{subfigure}
\usepackage{color}
\usepackage{url}
\usepackage{multirow}
\usepackage{amsfonts,amssymb}
\usepackage[amsmath,thmmarks]{ntheorem}




\usepackage{booktabs}
\usepackage{threeparttable}

\DeclareMathOperator{\orth}{orth}
\IEEEoverridecommandlockouts

\hyphenation{op-tical net-works semi-conduc-tor}


\begin{document}

\title{A Survey Of Attribute-based Encryption Schemes}


\author{Qxx% <-this % stops a space
\thanks{this is thanks}
}% <-this % stops a space
%\thanks{J. Doe and J. Doe are with Anonymous University.}% <-this % stops a space
%\thanks{Manuscript received April 19, 2005; revised August 26, 2015.}}





%\markboth{Journal of \LaTeX\ Class Files,~Vol.~14, No.~8, August~2015}%
%{Shell \MakeLowercase{\textit{et al.}}: Bare Demo of IEEEtran.cls for IEEE Journals}
% The only time the second header will appear is for the odd numbered pages
% after the title page when using the twoside option.

\maketitle


\begin{abstract}
The abstract goes here.
\end{abstract}

\begin{IEEEkeywords}
The keywords goes here.
\end{IEEEkeywords}

\IEEEpeerreviewmaketitle

\input introd.tex

\input defin.tex

\input state.tex

\input design_philosophy.tex

\input  extens.tex

\input  related.tex

\input  future.tex



\section{Conclusion}\label{sec:conclusion}
The conclusion goes here.






%\appendices
%\section{Proof of the First Zonklar Equation}
%Appendix one text goes here.
%
%\section{}
%Appendix two text goes here.
%
%\section*{Acknowledgment}
%
%The authors would like to thank...



\bibliographystyle{IEEEtran}
\bibliography{../bib/uestc_abe}

% biography section
\begin{IEEEbiography}{aaa}
Biography text here.
\end{IEEEbiography}

% if you will not have a photo at all:
\begin{IEEEbiography}{bbb}
Biography text here.
\end{IEEEbiography}

% insert where needed to balance the two columns on the last page with
% biographies
%\newpage

\begin{IEEEbiography}{ccc}
Biography text here.
\end{IEEEbiography}


\end{document}

